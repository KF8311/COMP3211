%%%%%%%%%%%%%%%%%%%%%%%%%%%%%%%%%%%%%%%%%%%%%%%%%%%%%%%%%%%%%%%%%%
%%%%%%%%%%%%%%%%%%%%%%%%%%%%%%%%%%%%%%%%%%%%%%%%%%%%%%%%%%%%%%%%%%
%Packages
\documentclass[10pt, a4paper]{article}
\usepackage[top=3cm, bottom=4cm, left=3.5cm, right=3.5cm]{geometry}
\usepackage{amsmath,amsthm,amsfonts,amssymb,amscd, fancyhdr, color, comment, graphicx, environ}
\usepackage{float}
\DeclareMathAlphabet{\mathscr}{U}{rsfs}{m}{n}
\usepackage[math-style=ISO]{unicode-math}
\setmathfont{TeX Gyre Termes Math}
\usepackage{lastpage}
\usepackage[dvipsnames]{xcolor}
\usepackage[framemethod=TikZ]{mdframed}
\usepackage{enumerate}
\usepackage[shortlabels]{enumitem}
\usepackage{fancyhdr}
\usepackage{indentfirst}
\usepackage{listings}
\usepackage{sectsty}
\usepackage{thmtools}
\usepackage{shadethm}
\usepackage{hyperref}
\usepackage{setspace}
\hypersetup{
    colorlinks=true,
    linkcolor=blue,
    filecolor=magenta,      
    urlcolor=blue,
}
%%%%%%%%%%%%%%%%%%%%%%%%%%%%%%%%%%%%%%%%%%%%%%%%%%%%%%%%%%%%%%%%%%
%%%%%%%%%%%%%%%%%%%%%%%%%%%%%%%%%%%%%%%%%%%%%%%%%%%%%%%%%%%%%%%%%%
%Environment setup
\mdfsetup{skipabove=\topskip,skipbelow=\topskip}
% \newrobustcmd\ExampleText{%An \textit{inhomogeneous linear} differential equation has the form
% \begin{align}
% L[v ] = f,
% \end{align}
% where $L$ is a linear differential operator, $v$ is the dependent
% variable, and $f$ is a given non−zero function of the independent
% variables alone.
% }
\mdfdefinestyle{theoremstyle}{%
linecolor=black,linewidth=1pt,%
frametitlerule=true,%
frametitlebackgroundcolor=gray!20,
innertopmargin=\topskip,
}
\newmdtheoremenv[style=theoremstyle]{Problem}{Problem}
\newenvironment{Solution}{\textbf{Solution.}}

\definecolor{codegreen}{rgb}{0,0.6,0}
\definecolor{codegray}{rgb}{0.5,0.5,0.5}
\definecolor{codepurple}{rgb}{0.58,0,0.82}
\definecolor{backcolour}{rgb}{0.95,0.95,0.92}

\lstdefinestyle{mystyle}{
    backgroundcolor=\color{backcolour},   
    commentstyle=\color{codegreen},
    keywordstyle=\color{magenta},
    numberstyle=\tiny\color{codegray},
    stringstyle=\color{codepurple},
    basicstyle=\ttfamily\footnotesize,
    breakatwhitespace=false,         
    breaklines=true,                 
    captionpos=b,                    
    keepspaces=true,                 
    numbers=left,                    
    numbersep=5pt,                  
    showspaces=false,                
    showstringspaces=false,
    showtabs=false,                  
    tabsize=2
}

\lstset{style=mystyle}
%%%%%%%%%%%%%%%%%%%%%%%%%%%%%%%%%%%%%%%%%%%%%%%%%%%%%%%%%%%%%%%%%%
%%%%%%%%%%%%%%%%%%%%%%%%%%%%%%%%%%%%%%%%%%%%%%%%%%%%%%%%%%%%%%%%%%
%Fill in the appropriate information below
\newcommand{\norm}[1]{\left\lVert#1\right\rVert}     
\newcommand\course{COMP3211}                            % <-- course name   
\newcommand\hwnumber{ 1}                                 % <-- homework number
\newcommand\Information{LAM, Pak Ho}                        % <-- personal information
%%%%%%%%%%%%%%%%%%%%%%%%%%%%%%%%%%%%%%%%%%%%%%%%%%%%%%%%%%%%%%%%%%
%%%%%%%%%%%%%%%%%%%%%%%%%%%%%%%%%%%%%%%%%%%%%%%%%%%%%%%%%%%%%%%%%%
%Page setup
\pagestyle{fancy}
\headheight 35pt
\lhead{\today}
\rhead{\includegraphics[width=2.5cm]{../img/logo-hkust.png}}
\lfoot{}
\pagenumbering{arabic}
\cfoot{\small\thepage}
\rfoot{}
\headsep 1.2em
\renewcommand{\baselinestretch}{1.25}
%%%%%%%%%%%%%%%%%%%%%%%%%%%%%%%%%%%%%%%%%%%%%%%%%%%%%%%%%%%%%%%%%%
%%%%%%%%%%%%%%%%%%%%%%%%%%%%%%%%%%%%%%%%%%%%%%%%%%%%%%%%%%%%%%%%%%
%Add new commands here
\renewcommand{\labelenumi}{\alph{enumi})}
\newcommand{\Z}{\mathbb Z}
\newcommand{\R}{\mathbb R}
\newcommand{\Q}{\mathbb Q}
\newcommand{\NN}{\mathbb N}
\newcommand{\PP}{\mathbb P}
\DeclareMathOperator{\Mod}{Mod} 
\renewcommand\lstlistingname{Algorithm}
\renewcommand\lstlistlistingname{Algorithms}
\def\lstlistingautorefname{Alg.}
\newtheorem*{theorem}{Theorem}
\newtheorem*{lemma}{Lemma}
\newtheorem{case}{Case}
\newcommand{\assign}{:=}
\newcommand{\infixiff}{\text{ iff }}
\newcommand{\nobracket}{}
\newcommand{\backassign}{=:}
\newcommand{\tmmathbf}[1]{\ensuremath{\boldsymbol{#1}}}
\newcommand{\tmop}[1]{\ensuremath{\operatorname{#1}}}
\newcommand{\tmtextbf}[1]{\text{{\bfseries{#1}}}}
\newcommand{\tmtextit}[1]{\text{{\itshape{#1}}}}

\newenvironment{itemizedot}{\begin{itemize} \renewcommand{\labelitemi}{$\bullet$}\renewcommand{\labelitemii}{$\bullet$}\renewcommand{\labelitemiii}{$\bullet$}\renewcommand{\labelitemiv}{$\bullet$}}{\end{itemize}}
\catcode`\<=\active \def<{
\fontencoding{T1}\selectfont\symbol{60}\fontencoding{\encodingdefault}}
\catcode`\>=\active \def>{
\fontencoding{T1}\selectfont\symbol{62}\fontencoding{\encodingdefault}}
\catcode`\<=\active \def<{
\fontencoding{T1}\selectfont\symbol{60}\fontencoding{\encodingdefault}}

%%%%%%%%%%%%%%%%%%%%%%%%%%%%%%%%%%%%%%%%%%%%%%%%%%%%%%%%%%%%%%%%%%
%%%%%%%%%%%%%%%%%%%%%%%%%%%%%%%%%%%%%%%%%%%%%%%%%%%%%%%%%%%%%%%%%%
%Begin now!



\begin{document}

\begin{titlepage}
    \begin{center}
        \vspace*{3cm}

        \Huge
        \textbf{COMP3211}

        \vspace{1cm}
        \huge
        Homework\hwnumber

        \vspace{1.5cm}
        \Large

        \textbf{\Information}                      % <-- author


        \vfill

        A \course \ Written Assignment

        \vspace{1cm}

        \includegraphics[width=0.4\textwidth]{../img/logo-hkust.png}
        \\

        \Large

        \today

    \end{center}
\end{titlepage}

%%%%%%%%%%%%%%%%%%%%%%%%%%%%%%%%%%%%%%%%%%%%%%%%%%%%%%%%%%%%%%%%%%
%%%%%%%%%%%%%%%%%%%%%%%%%%%%%%%%%%%%%%%%%%%%%%%%%%%%%%%%%%%%%%%%%%
%Start the assignment now
%%%%%%%%%%%%%%%%%%%%%%%%%%%%%%%%%%%%%%%%%%%%%%%%%%%%%%%%%%%%%%%%%%
%New problem
\newpage
\begingroup\huge{\textbf{Question 1}}
\endgroup \\
1.1. No, we have the 4 sensors
of $s_2, s_4, s_6, s_8$, and such agent can be designed as
\begin{flalign*}
    s_2 \bar{s_4} & \rightarrow \text{east}  \\
    s_4 \bar{s_6} & \rightarrow \text{south} \\
    s_6 \bar{s_8} & \rightarrow \text{west}  \\
    s_8 \bar{s_2} & \rightarrow \text{north} \\
    1             & \rightarrow \text{north}
\end{flalign*}
When there is a wall above the robot and the east side is cleared,
we will move east. When there is a wall in the east side and the south side is cleared,
we will move south.
When there is a wall in the south side and the west side is cleared,
we will move west.
When there is a wall in the west side and the north side is cleared,
we will move north. And if theres no wall around we are default to
move the robot to north.
However, this rule set fails because there exist position where the local pattern $s_2, s_4, s_6, s_8$
is equal but the actual action should be different. Consider a case
such that the robot walk around the inner obstacle, when it walk
through the corners all sensors will return 0. Since in the actual
case only $s_1$, $s_3$, $s_5$ or $s_7$ is 1, in the local sensors,
all these position is returned as all 0. A memoryless agent seeing (0,0,0,0)
at all corners must pick the same action (north),
which works at one corner but fails at the other three. So all positon will lead
to the action North, however, if we want to make the robot
follow the rectangle outer boundary located in the middle of the map,
they should perform east, west, south or north respectively,
so the local sensors cannot distinguish these positions. \\

1.2 We cannot "compute" $s_7$ out. Normally,
with memory we can set $w_7  = 1$ iff in the previous step $w_8=1$
and the robot move north previously. We have the design
\begin{flalign*}
    s_2 \bar{s_4} & \rightarrow \text{east}  \\
    s_4 \bar{s_6} & \rightarrow \text{south} \\
    s_6 \bar{s_8} & \rightarrow \text{west}  \\
    s_8 \bar{s_2} & \rightarrow \text{north} \\
    w_3           & \rightarrow \text{east}  \\
    w_5           & \rightarrow \text{south} \\
    w_7           & \rightarrow \text{west}  \\
    w_1           & \rightarrow \text{north} \\
    1             & \rightarrow \text{north}
\end{flalign*}
With:
\begin{itemize}
    \item $w_1 = 1$ \quad iff at the previous time step, $w_2 = 1$, and the robot moved east;
    \item $w_3 = 1$ \quad iff at the previous time step, $w_4 = 1$, and the robot moved south;
    \item $w_5 = 1$ \quad iff at the previous time step, $w_6 = 1$, and the robot moved west;
    \item $w_7 = 1$ \quad iff at the previous time step, $w_8 = 1$, and the robot moved north.
\end{itemize}
However, notice that this robot moves in clockwise way in outer boundary,
such that when we approach the southeast corner, such that $s_5$ is true.
But, in the actual case if the outer boundary is a simple rectangle,
$s_3$, $s_5$ and $s_7$  are all 1, which is different from what we predict.
The $w_7$ we are predicting are only used for tracking the outer boundary,
so that when we reach the southwest corner we will go north, which does
not computer $s_7$ in any situations.

\newpage
\begingroup\huge{\textbf{Question 2}}
\endgroup \\
2.1. Consider all possible cases:
We notices that when $x_3$ is 1 as long as $x_2$ is 0 then
the TLU will reach a value of 3 or above regardless of other input,
which give us the first term $x_3 \land \bar{x_2}$
and when $x_3$ is 1 or any of $x_1$ or $x_4$ is 1, then it will also be
more than 3 regardless of values of $x_2$ and $x_5$ (even both of them are 1)
the value computed is more than 3. Which give us the second term $x_3\land(x_1 \lor x_4)$.
Lastly, when only $x_1$ and $x_4$ is 1, $x_5$ is acceptable to be 1
since its value is larger than 3 but $x_2$ must not be 1 as it will
lower the value to lower than 3, so the third term is $(x_1\land x_4)\land \bar{x_2}$
Combining the term, the boolean function is:
$$f=\bar{x_2}\land(x_3 \lor (x_1 \land x_4)) \lor (x_3 \land (x_1 \lor x_4))$$

2.2 The question ask for boolean function which threshold is larger or equal
to 10. However, when we evaluate the maximum of the TLU, which
we assign value 1 to all positive values and assign value 0 to all negative
values, we compute: $1.9\times 1 -2.5 \times 0 +5.2\times 1 +2.3\times 1-0.5 \times 0= 9.4 <10$.
As the max value is also small than 10, any inputs of this TLU will output
a boolean result of false, we can directly state that the boolean function is
$$f=\text{false}$$
%%%%%%%%%%%%%%%%%%%%%%%%%%%%%%%%%%%%%%%%%%%%%%%%%%%%%%%%%%%%%%%%%%
%Complete the assignment now
\end{document}

%%%%%%%%%%%%%%%%%%%%%%%%%%%%%%%%%%%%%%%%%%%%%%%%%%%%%%%%%%%%%%%%%%
%%%%%%%%%%%%%%%%%%%%%%%%%%%%%%%%%%%%%%%%%%%%%%%%%%%%%%%%%%%%%%%%%%
